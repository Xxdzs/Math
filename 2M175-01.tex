\documentclass[12pt,a4paper,final]{article}

\usepackage[french]{babel}

% ---== Standard packages ==---
\usepackage[utf8]{inputenc}
\usepackage{amsmath}
\usepackage{amsfonts}
\usepackage{amssymb}
\usepackage{graphicx}
\usepackage{lmodern}
% ---== Extra packages ==---
\usepackage{stmaryrd}
\usepackage{relsize}
\usepackage{amsthm}
\usepackage{mathrsfs}

% symmetry symetrie
% personal personnel

% ---== Personal aliases ==---
\newcommand{\N}{\ensuremath{\mathbb{N}}}
\newcommand{\Z}{\ensuremath{\mathbb{Z}}}
\newcommand{\R}{\ensuremath{\mathbb{R}}}
\newcommand{\K}{\ensuremath{\mathbb{K}}}
\newcommand{\B}{\ensuremath{\mathcal{B}}}
\newcommand{\A}{\ensuremath{\mathscr{A}}}
\newcommand{\Sn}{\ensuremath{\mathfrak{S}_{n}}}

% ---== Theorems ==--- 
\newtheorem{theorem}{Th\'eoreme}
\theoremstyle{definition}
\newtheorem{defn}{D\'efinition}
\newtheorem{prop}{Proposition}
\theoremstyle{remark}
\newtheorem{rem}{Remarque}
\newtheorem{exmpl}{Exemple}

% ---== Edits ==---
\DeclareMathOperator{\Det}{det}
\DeclareMathOperator{\id}{Id}
\renewcommand{\thesubsection}{\arabic{subsection}}

% ---== Personal commands ==---
\newcommand{\gen}[1]{\ensuremath{\langle#1\rangle}}
\newcommand{\either}[3]{\ensuremath{\left#1\begin{matrix}#2\\#3\end{matrix}\right.}}
\newcommand{\eucldiv}[4]{\ensuremath{
	\exists(#3,#4)\in\Z\times\N,\either\{{#1 = #2 #3 + #4}{0\leqslant #4 < #2}
}}
\newcommand{\fun}[5]{\ensuremath{
	\begin{matrix}
		#1 :	& #2 & \to		& #3 \\
				& #4 & \mapsto	& #1(#4) = #5
	\end{matrix}
}}


\author{Antoine Ducros}
\title{Groupes et morphismes}

\begin{document}
	\maketitle
%	\tableofcontents
%	\newpage
	\section{Exemples de sous-groupes}
		\subsection{Sous-groupes de $\mathfrak{S}_4$}
			\begin{enumerate}
			\item Soit $K$ le sous-ensemble $\{\id, (12)(34), (13)(24), (14)(23) \}$.
				C'est un sous-groupe de $(\mathfrak{S}_4, \circ)$ :
				\begin{description}
					\item[Non vide et \'el\'ement neutre] $\id\in K$
					\item[Loi interne] $K$ est stable par produit, en effet
						\begin{align*}
							&\forall \sigma \in K, \sigma^2 = \id \\
							(12)(34)\circ(13)(24) &= (13)(24)\circ(12)(34) = (14)(23) \\
							(12)(34)\circ(14)(23) &= (14)(23)\circ(12)(34) = (13)(24) \\
							(13)(24)\circ(14)(23) &= (14)(23)\circ(13)(24) = (12)(34)
						\end{align*}
				\end{description}
			\item Le sous-groupe engendr\'e par $(12)$ est $\{(12)^n / n\in\N\}$.
				Or $(12)^2=\id$ donc
				\[ \gen{(12)} ~ =\{\id, (12)\} \]
			\item On sait que $(1234)^4=\id$, donc
			\begin{align*}
				\gen{(1234)} &= \{ \id, (1234), (1234)^2 , (1234)^3 \} \\
				(1234)^3 &= (1234)^{-1} = (4321) \\
				(1234)^2 &=
					\left(\begin{matrix}
						1~2~3~4 \\
						3~4~1~2
					\end{matrix}\right)
					= (13)(24)
			\end{align*}
			Le groupe est donc :
			\[ \gen{(1234)} = \{\id,(1234),(13)(24),(4321)\} \]
			\end{enumerate}
		\subsection{Sous-groupes de \Z}
			Soit $d\in\Z, \gen{d}=\{nd|n\in\Z\}$ (notation additive) On le note $d\Z$
		\begin{theorem}
			Tout sous-groupe de \Z~est de la forme $d\Z$, pour un unique $d\in\N$
		\end{theorem}
		\begin{proof}
			Soit $G$ un sous-groupe de $\Z$. Montrons qu'il existe $d\in\N$, tel que $G=d\Z$.
			\begin{description}
				\item[Existence] On distingue deux cas
				\begin{itemize}
					\item[•] Si $G=\{0\}, G=0\Z : d\leftarrow0$ convient.
					\item[•] Si $G$ est non trivial, il existe $g\neq0$ dans $G$, avec $-g\in G$.
						$G$ contient donc au moins un \'el\'ement strictement positif, d'o\`u
						$G\cap\N^*$  est une partie non vide de \N, donc elle admet un plus petit \'el\'ement $d$.
						$d\in G \Rightarrow \gen{d}\subset G$. Nous allons \'etablir l'inclusion r\'eciproque.

						Soit $g\in G$, effectuons la division euclidienne de $g$ par $d$ :
						\[ \eucldiv gdqr \]
						\[ \Rightarrow r = \underbrace{dq}_{\in d\Z \subset G}  \underbrace{-g}_{\in G} \in G \]
						\[ \left. \begin{matrix}
								0 \leqslant r < d \\
								r \in G \\
								d=\min ( G \cap \N^* )
							\end{matrix} \right\}
							\Rightarrow r = 0 \Rightarrow g = dq
						\]
						Ainsi $G \subset d\Z$. \\
						On a donc bien $G=d\Z$
				\end{itemize}
				\item[Unicit\'e]
					Soit $d,d'\in\N$, tels que $d\Z=d'\Z$
					\begin{itemize}
						Si $d=0$ alors $d\Z=0$ donc $d'\Z=0$ donc $d'=0$ car $d'=d'\times1\in d'\Z$
					\end{itemize}
			\end{description}
		\end{proof}
		
\end{document}
