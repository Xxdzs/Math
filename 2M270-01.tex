\documentclass[12pt,a4paper,final]{article}

\usepackage[french]{babel}

% ---== Standard packages ==---
\usepackage[utf8]{inputenc}
\usepackage{amsmath}
\usepackage{amsfonts}
\usepackage{amssymb}
\usepackage{graphicx}
\usepackage{lmodern}
% ---== Extra packages ==---
\usepackage{stmaryrd}
\usepackage{relsize}
\usepackage{amsthm}
\usepackage{mathrsfs}

% symmetry symetrie
% personal personnel

% ---== Personal aliases ==---
\newcommand{\N}{\ensuremath{\mathbb{N}}}
\newcommand{\Z}{\ensuremath{\mathbb{Z}}}
\newcommand{\R}{\ensuremath{\mathbb{R}}}
\newcommand{\K}{\ensuremath{\mathbb{K}}}
\newcommand{\B}{\ensuremath{\mathcal{B}}}
\newcommand{\A}{\ensuremath{\mathscr{A}}}
\newcommand{\Sn}{\ensuremath{\mathfrak{S}_{n}}}

% ---== Theorems ==--- 
\newtheorem{theorem}{Th\'eoreme}
\theoremstyle{definition}
\newtheorem{defn}{D\'efinition}
\newtheorem{prop}{Proposition}
\theoremstyle{remark}
\newtheorem{rem}{Remarque}
\newtheorem{exmpl}{Exemple}

\DeclareMathOperator{\Det}{det}

% ---== Personal commands ==---
\newcommand{\either}[3]{\ensuremath{\left#1\begin{matrix}#2\\#3\end{matrix}\right.}}
\newcommand{\fun}[5]{\ensuremath{
	\begin{matrix}
		#1 :	& #2 & \to		& #3 \\
				& #4 & \mapsto	& #1(#4) = #5
	\end{matrix}
}}


\author{Vincent Humili\`ere}
\title{Formes $n$-lin\'eaires antisym\'etriques\\sur un espace de dimension $n$}

\begin{document}
	\maketitle
	\noindent
	Soit $n \in \N$, \\
	\phantom{Soit} $E$ un \K -espace vectoriel de dimension $n$, \\
	\phantom{Soit} $\B = (e_1, ..., e_n)$ une base de E, \\
	\phantom{Soit} $\alpha : E^n \to \K$ une forme $n$-lin\'eaire antisym\'etrique.

	\section*{Rappels}
		\subsection*{$\alpha$ est $n$-lin\'eaire :}
			\[ \forall (a, b, c_1, ..., c_{n-1}) \in E^{n+1}, \forall \lambda \in \K, \]
			\[
				\alpha(c_1, ..., \lambda a+b, ..., c_{n-1})
				= \lambda \alpha(c_1, ..., a, ..., c_{n-1})
				+ \alpha(c_1, ..., b, ..., c_{n-1})
			\]
		\subsection*{$\alpha$ est antisym\'etrique :}
			\[ \forall (a_1, ..., a_n) \in E^n, 1 \leqslant i, j \leqslant n, i \neq j\]
			\[ \alpha(a_1, ..., a_j, ..., a_i, ..., a_n) = -\alpha(a_1, ..., a_i, ..., a_j, ..., a_n)\]

	\newpage
	\noindent
	Soit $(v_1, ..., v_n) \in E^n$, \\
	On note, $\forall i \in \llbracket 1, n \rrbracket, v_i = \sum\limits_{j=1}^n v_{ij}e_j$
	\begin{align*}
		\alpha(v_1, ..., v_n)
			&= \alpha(\sum_{j=1}^nv_{1j}e_j, v_2, ..., v_n) \\
		\text{\tiny{(Lin\'earit\'e pour $v_1$)}}
			&= \sum_{j=1}^nv_{1j}\alpha(e_j, v_2, ..., v_n) \\
		&= \sum_{j=1}^nv_{1j}\alpha(e_j, \sum_{k=1}^nv_{2k}e_k, v_3, ..., v_n) \\
		\text{\tiny{(Lin\'earit\'e pour $v_2$)}}
			&= \sum_{j=1}^n\sum_{k=1}^nv_{1j}v_{2k}\alpha(e_j, e_k, v_3, ..., v_n) \\
		\text{On note} j_1, j_2, j_3, & \ldots \text{ au lieu de } j, k, l \ldots \\
		\sum_{j_1=1}^n\sum_{j_2=1}^nv_{1j_1}v_{2j_2}\alpha(e_{j_1}, e_{j_2}, v_3, ..., v_n)
			&= \sum_{j_1=1}^n\ldots\sum_{j_n=1}^nv_{1j_1}\ldots v_{nj_n}\alpha(e_{j_1}, ..., e_{j_n})
	\end{align*}
	Or $\alpha$ est antisym\'etrique, donc $\alpha(e_{j_1}, ..., e_{j_n}) = 0$ s'il existe deux $j_k$ et $j_l$ \'egaux.
	Donc $\alpha(e_{j_1}, \ldots)$ ne peut \^etre non nul que si tous les $j_k$ sont distincts, \\
	\textit{i.e.} si
	$\begin{matrix}
		\{1, ..., n\} & \to & \{1, ..., n\} \\
		i & \mapsto & j_i
	\end{matrix}$
	est une bijection.
	\begin{align*}
		\text{Donc } \alpha(v_1, ..., v_n)
			&= \sum_{\underset{\underset{\text{une bijection}}{\text{o\`u }
			i\mapsto j_i \text{ est}}}{j_1, ..., j_n = 1}}^n
			v_{1j_1} \ldots v_{nj_n} \alpha(e_{j_1}, ..., e_{j_n}) \\
		&= \sum_{\sigma \in \Sn}v_{1\sigma(1)} \ldots v_{n\sigma(n)}
			\alpha(e_{\sigma(1)}, ..., e_{\sigma(n)})
	\end{align*}
	Ainsi, puisque $\alpha$ est antisym\'etrique,
	\begin{equation} \label{eq:1}
		\boxed{
			\alpha(v_1, ..., v_n) =
			\left( \sum_{\sigma\in\Sn}\varepsilon(\sigma)v_{1\sigma(1)} \ldots v_{n\sigma(n)} \right)
			\alpha(e_1, ..., e_n)
		}
	\end{equation}
	\newpage
	\begin{defn}
		On appelle d\'eterminant des vecteurs $v_1, ..., v_n$ dans la base $\B=(e_1, ..., e_n)$ le scalaire
		\[ \Det_\B (v_1, ..., v_n) = \sum_{\sigma\in\Sn}\varepsilon(\sigma)v_{1\sigma(1)} \ldots v_{n\sigma(n)} \]
	\end{defn}
	\begin{prop}
		Pour toute forme $n$-lin\'eaire $\alpha$ sur $E$ (o\`u dim$(E) = n$)
		il existe un scalaire $\lambda \in \K$ tel que $\alpha=\lambda\Det_\B$. \\
		De plus $\lambda=\alpha(\B)$.
	\end{prop}
	\begin{proof}
		En effet
		\[ \alpha(v_1, ..., v_n) = \underbrace{\alpha(e_1, ..., e_n)}_{\alpha(\B)}\Det_\B(v_1, ..., v_n) \]
		d'apr\`es l'\'equation (\ref{eq:1}).
	\end{proof}
	\begin{prop}
		$\Det_\B$ est $n$-lin\'eaire antisym\'etrique.
	\end{prop}
	\begin{proof}
		Sera faite la fois prochaine.
	\end{proof}
	\begin{prop}
		Soit $\B, \B'$ deux bases de $E$\\
		Pour tous vecteurs $v_1, ..., v_n$ dans $E$,
		\[ \Det_{\B'}(v_1, ..., v_n) = \Det_{\B'}(\B) \Det_\B(v_1, ..., v_n)\]
	\end{prop}
	\section*{Interpr\'etation g\'eom\'etrique}
		\subsection*{Dans $\R^2$}
			\noindent
			$\B_0 = (e_1, e_2)$ base canonique, \\
			$u,v,u_1,u_2 \in \R^2$, \\
			$\A(u, v)$ l'aire orient\'ee du parall\`elogramme engendr\'e par $u$ et $v$. \\
			On d\'etermine graphiquement que :
			\begin{align*}
				\A(v,u)				&= -\A(u,v) \\
				\A(u_1+u_2)			&= \A(u_1,v)+\A(u_2,v) \\
				\A(\lambda u, v)	&= \lambda\A(u,v)
			\end{align*}
			Donc \A est une forme bilin\'eaire antisym\'etrique, \textit{i.e.} est un multiple du d\'eterminant :
			\[ \A(u,v) = \A(e_1,e_2)\Det_{\B_0}(u,v) \]
			Or $\A(e_1, e_2)=1$, d'o\`u :
			\[\boxed{
				\A(u,v) = \Det_{\B_0}(u,v)
			}\]
		\subsection*{De m\^eme dans $\R^3$}
			$\Det_{\B_0}(u,v,w)$ est le volume orient\'e du parall\'el\'epip\`ede engendr\'e par u,v,w.
\end{document}
